\documentclass{beamer}

% 主题设置
\usetheme{CambridgeUS}
\usepackage{xeCJK}
\setCJKmainfont{SimSun}
\usepackage{graphicx}
\usepackage{subfigure}
\usepackage{amsmath}
\usepackage{amssymb}

% 页面信息
\title{动力系统模型及其在实时血糖预测中的应用}
\author{刘益通}
\institute{上海科技大学}
\date{\today}
\logo{\includegraphics[height=0.8cm]{Img/ShanghaiTech_Logo.png}}
\begin{document}

% 标题页
\begin{frame}
  \titlepage
\end{frame}

% 目录页
\begin{frame}
  \frametitle{目录}
  \tableofcontents
\end{frame}

% 第一部分
\section{研究背景}
\begin{frame}
  \frametitle{研究背景}
    WHO数据显示,全球有4.25亿糖尿病患者,占全球人口的8.5\%,而且这个数字还在不断增长。糖尿病问题已经成为全球性的公共卫生问题。CGM技术的发展为糖尿病患者提供了一种实时监测血糖的方法,但是如何利用这些数据进行血糖预测仍然是一个挑战。
\end{frame}

\begin{frame}
  \frametitle{研究目标}
  
\end{frame}

% 第二部分
\section{相关工作}
\begin{frame}
  \frametitle{文献综述}
  \begin{itemize}
    \item 动力系统模型基础理论
    \item 现有血糖预测模型
    \item 动力系统在其他领域的应用
  \end{itemize}
\end{frame}

% 第三部分
\section{方法}
\begin{frame}
  \frametitle{研究方法}
  \begin{itemize}
    \item 动力系统模型的建立
    \item 模型参数的确定
    \item 实时预测算法设计
  \end{itemize}
\end{frame}

\begin{frame}
  \frametitle{模型构建}
  \begin{itemize}
    \item 差分方程的选择
    \item 模型变量及其关系
    \item 初始条件的设置
  \end{itemize}
\end{frame}

% 第四部分
\section{实验与结果}
\begin{frame}
  \frametitle{实验设计}
  \begin{itemize}
    \item 实验数据来源
    \item 数据预处理方法
    \item 实验流程
  \end{itemize}
\end{frame}

\begin{frame}
  \frametitle{实验结果}
  \begin{itemize}
    \item 预测精度分析
    \item 实时性分析
    \item 模型性能比较
  \end{itemize}
\end{frame}

\begin{frame}
  \frametitle{结果展示}
\end{frame}

% 第五部分
\section{结论}
\begin{frame}
  \frametitle{研究结论}
  \begin{itemize}
    \item 总结主要研究发现
    \item 模型的优点与不足
    \item 对未来研究的建议
  \end{itemize}
\end{frame}

% 致谢
\begin{frame}
  \frametitle{致谢}
  \begin{itemize}
    \item 感谢导师的指导
    \item 感谢同学的帮助
    \item 感谢家人的支持
  \end{itemize}
\end{frame}

\end{document}
