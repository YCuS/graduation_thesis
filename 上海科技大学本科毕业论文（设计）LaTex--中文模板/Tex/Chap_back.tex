\chapter{数学基础}\label{chap:background}
\section{矩阵特征值和特征向量}
\begin{defn}[特征值和特征向量]
    给定一个方阵 \( A \in \mathbb{R}^{n \times n} \),特征值 \( \lambda \) 和特征向量 \( \mathbf{v} \in \mathbb{R}^n \) 满足以下方程:

\[
A \mathbf{v} = \lambda \mathbf{v}
\]

其中 \( \mathbf{v} \neq 0 \) 是特征向量,\( \lambda \) 是对应的特征值。
\end{defn}


\subsection{求解方法}

先给出矩阵的特征方程

    \[
    \det(A - \lambda I) = 0
    \]

其中 \( \det \) 表示行列式,\( I \) 是单位矩阵。展开特征方程,得到一个 \( n \) 次多项式方程。通过求解这个方程,得到矩阵 \( A \) 的特征值 \( \lambda \)。 对于每一个特征值 \( \lambda \),求解以下齐次线性方程组:

    \[
    (A - \lambda I) \mathbf{v} = 0
    \]

这个方程组的解即为对应特征值 \( \lambda \) 的特征向量 \( \mathbf{v} \)。

另外,矩阵的迹和行列式也与特征值有关
\begin{defn}
    对于一个$n \times n$的矩阵
    \begin{equation}
        \mathbf{A}=\begin{pmatrix}
            a_{11} & a_{12} & \cdots & a_{1n} \\
            a_{21} & a_{22} & \cdots & a_{2n} \\
            \vdots & \vdots & \ddots & \vdots \\
            a_{n1} & a_{n2} & \cdots & a_{nn}
        \end{pmatrix}
    \end{equation}
    其迹定义为
    \begin{equation}
        \text{tr}(\mathbf{A}) = \sum_{i=1}^{n} a_{ii}
    \end{equation}
\end{defn}
\begin{thm}\label{thm:trace_det}
    矩阵的迹等于其特征值之和,矩阵的行列式等于其特征值之积
    \begin{equation}
        \text{tr}(\mathbf{A}) = \sum_{i=1}^{n} \lambda_i,\quad \det(\mathbf{A}) = \prod_{i=1}^{n} \lambda_i
    \end{equation}
    其中$\lambda_i$是矩阵$\mathbf{A}$的特征值
\end{thm}
\begin{pf}
   对于矩阵$\mathbf{A}= \begin{pmatrix}
    a_{11} & a_{12} & \cdots & a_{1n} \\
    a_{21} & a_{22} & \cdots & a_{2n} \\
    \vdots & \vdots & \ddots & \vdots \\
    a_{n1} & a_{n2} & \cdots & a_{nn}
\end{pmatrix}$,其特征方程为
\begin{equation}\label{eq:char_eq}
    f(\lambda)=\det(\mathbf{A} - \lambda \mathbf{I}) = \left| \begin{array}{cccc}
        a_{11}-\lambda & a_{12} & \cdots & a_{1n} \\
        a_{21} & a_{22}-\lambda & \cdots & a_{2n} \\
        \vdots & \vdots & \ddots & \vdots \\
        a_{n1} & a_{n2} & \cdots & a_{nn}-\lambda
    \end{array} \right| = 0
\end{equation}
特征多项式的常数项为$f(0)=\det(\mathbf{A})$,即矩阵的行列式。
对$\mathbf{A}$做Jordan标准型分解,得到$\mathbf{A}= \mathbf{P} \mathbf{J} \mathbf{P}^{-1}$,其中$\mathbf{J}$是Jordan标准型矩阵,$\mathbf{P}$是可逆矩阵。由于$\mathbf{J}$是上三角矩阵,其对角元素即为特征值,所以
\begin{equation}
    \text{tr}(\mathbf{A})=\text{tr}(\mathbf{PJP^{-1}})=\text{tr}(\mathbf{PP^{-1}J})=\text{tr}(\mathbf{J})=\sum_{i=1}^{n} \lambda_i。
\end{equation}
可知矩阵的迹等于其特征值之和。
\end{pf}
\section{巴拿赫(Banach)压缩映射定理}
\begin{defn}[压缩映射]\label{def:comp_map}
    对于映射
\begin{align*}
    f : (X,d) & \to (X,d) \\
    x & \mapsto f(x)
\end{align*}
其中,$(X,d)$ 是一个以$d$为度量的度量空间,$f$ 是一个映射。如果满足$\forall x, y \in X$,有
\begin{equation}
    d(f(x), f(y)) \leq k \cdot d(x, y)
\end{equation}
其中,$0 \leq k < 1$,则称$f$是一个压缩映射。
\end{defn}

\begin{thm}[巴拿赫压缩映射定理]\label{thm:banach}
    设 $(X, d)$ 是一个非空完备度量空间,$f : X \to X$ 是一个压缩映射。那么,存在唯一的不动点 $x^* \in X$,满足 $f(x^*) = x^*$。
\end{thm}
\begin{pf}
    对于$x_0 \in X$,定义序列$\{x_n\}$为$x_{n+1}=f(x_n)$,对于$\forall n \in \mathbb{N}$,有
    \begin{equation*}
        d(x_{n+1}, x_n)  \leq k^n \cdot d(x_1, x_0)
    \end{equation*}
    接下来证明$\{x_n\}$是一个柯西序列。对于$\forall m > n$,有
    \begin{align*}
        d(x_m, x_n) & = d(f(x_{m-1}), f(x_{n-1})) \\
        & \leq k^{m-1} \cdot d(x_1, x_0) + k^{m-2} \cdot d(x_1, x_0) + \cdots + k^{n} \cdot d(x_1, x_0) \\
        & = k^n \cdot d(x_1, x_0) \cdot \sum_{i=0}^{m-n-1} k^i \\
        & = k^n \cdot d(x_1, x_0) \cdot \frac{1-k^{m-n}}{1-k} \\
        & \leq k^n \cdot d(x_1, x_0) \cdot \frac{1}{1-k}
    \end{align*}
    对于$\forall \epsilon > 0$,取$n_0$使得$k^{n_0} \cdot d(x_1, x_0) \cdot \frac{1}{1-k} < \epsilon$,则对于$\forall m, n > n_0$,有$d(x_m, x_n) < \epsilon$,所以$\{x_n\}$是一个柯西序列。由于$(X, d)$是完备度量空间,所以$\{x_n\}$收敛,设其极限为$x^*$,则有
    \begin{equation*}
        x^* = \lim_{n \to \infty} x_n = \lim_{n \to \infty} f(x_{n-1}) = f(\lim_{n \to \infty} x_{n-1}) = f(x^*)
    \end{equation*}
    所以$x^*$是$f$的一个不动点。接下来证明$x^*$是唯一的不动点。设$x^*$和$y^*$都是$f$的不动点,则有
    \begin{equation*}
        d(x^*, y^*) = d(f(x^*), f(y^*)) \leq k \cdot d(x^*, y^*)
    \end{equation*}
    由于$0 \leq k < 1$,所以$d(x^*, y^*) = 0$,即$x^* = y^*$,所以$x^*$是唯一的不动点。
\end{pf}

\section{Grönwall不等式}

Grönwall不等式是一个给出函数上界的不等式,是证明一些微分方程问题的重要工具。
\begin{thm}[Grönwall不等式(积分形式)]
    令$I$是$[a,b],[a,b)$或$[a,+\infty)$形式的区间,$\alpha,\beta,u$是三个定义在$I$上的实值函数,其中$\beta$非负连续,$u$连续,$\alpha$的非负部分在$I$的任意有界闭子区间上可积,若对于$I$中的任意$t$,$u$都满足
    \begin{equation}\label{eq:gronwall}
        u(t) \leq \alpha(t) + \int_{a}^{t} \beta(s)u(s) ds
    \end{equation}
    那么
    \begin{equation}
        u(t) \leq \alpha(t) + \int_{a}^{t} \alpha(s)\beta(s) e^{\int_{s}^{t} \beta(r) dr} ds,\forall t \in I
    \end{equation}
    如果$\alpha$是一个不减函数,那么
    \begin{equation}\label{eq:gronwall2}
        u(t) \leq \alpha(t) e^{\int_{a}^{t} \beta(s) ds},\forall t \in I
    \end{equation}
\end{thm}
\begin{pf}
    我们引入一个函数$v(t)$,定义为
    \begin{equation}
        v(t) = e^{-\int_{a}^{t} \beta(s) ds}\int_{a}^{t} \beta(s) u(s)ds
    \end{equation}
    对$v(t)$求导,得到
    \begin{equation}\label{eq:gronwall_proof}
        \begin{aligned}
            v'(t) & = -\beta(t) e^{-\int_{a}^{t} \beta(s) ds}\int_{a}^{t} \beta(s) u(s)ds + e^{-\int_{a}^{t} \beta(s) ds} \beta(t) u(t)\\
            & = e^{-\int_{a}^{t} \beta(s) ds} \beta(t) (u(t)-\int_{a}^{t} \beta(s) u(s)ds)\\
            & \leq e^{-\int_{a}^{t} \beta(s) ds} \beta(t) \alpha(t)(\text{由式\ref{eq:gronwall}})
        \end{aligned}
    \end{equation}
    由于$\beta$非负,所以不等式\ref{eq:gronwall_proof}给出了$v(t)$的导数的上界。由于$v(a)=0$,对$v(s)$从$a$到$t$积分,得到
    \begin{equation}\label{eq:gronwall_proof2}
        \begin{aligned}
            v(t) & = \int_{a}^{t} v'(s) ds\\
            & \leq \int_{a}^{t} e^{-\int_{a}^{s} \beta(r) dr} \beta(s) \alpha(s) ds
        \end{aligned}
    \end{equation}
    由$v(t)$的定义,我们有
    \begin{equation}
        \begin{aligned}
            \int_a^t \beta(s) u(s) ds & = e^{\int_a^t \beta(s) ds} v(t),\text{由\ref{eq:gronwall_proof2}得}\\
            &\leq \int_a^t \alpha(s) \beta(s) e^{(\int_a^t \beta(r) dr-\int_{a}^{s} \beta(r) dr)} ds\\
            &= \int_a^t \alpha(s) \beta(s) e^{\int_s^t \beta(r) dr} ds
        \end{aligned}
    \end{equation}
    结合式\ref{eq:gronwall},我们有
    \begin{equation}
        u(t) \leq \alpha(t) + \int_a^t \alpha(s) \beta(s) e^{\int_s^t \beta(r) dr} ds
    \end{equation}
    如果$\alpha$是一个不减函数,那么$\alpha(s) \leq \alpha(t)$,所以
    \begin{equation}
        \begin{aligned}
            u(t) & \leq \alpha(t) + \int_a^t \alpha(s) \beta(s) e^{\int_s^t \beta(r) dr} ds\\
            & \leq \alpha(t) + \int_a^t \alpha(t) \beta(s) e^{\int_s^t \beta(r) dr} ds\\
            & = \alpha(t) +\alpha(t)\left[-e^{\int_s^t \beta(r)dr}\right]_{s=a}^{s=t}\\
            & = \alpha(t) e^{\int_a^t \beta(s) ds}
        \end{aligned}
    \end{equation}
\end{pf}
