\chapter{血糖预测中动力系统模型的发展}
\section{葡萄糖动力系统}
在后吸收状态下,葡萄糖由肝脏和肾脏释放到血液中,被体内所有细胞从间质液中移除,并分布到许多生理组分中(例如,动脉血、静脉血、脑脊液、间质液)。我们可以通过以下微分方程来描述葡萄糖动力系统\cite{bergman1979quantitative}:

\begin{equation}\label{1}
    \frac{dG}{dt} = \text{Production} - \text{Uptake},
\end{equation}
其中$G$是血液中的葡萄糖浓度,$t$是时间,Production是葡萄糖生成速率,Uptake是葡萄糖摄取速率(也可以理解为血液中葡萄糖的消耗速率)。

葡萄糖产生和摄取的速率主要取决于血糖和胰岛素水平。这些关系已经通过葡萄糖夹持技术进行了实验定义,该技术允许在各种稳态血糖和胰岛素水平下测量葡萄糖产生和摄取速率\cite{bergman1985assessment}。在恒定胰岛素水平下,葡萄糖产生减少而摄取增加,两者都与血糖水平线性相关\cite{best1981glucose}。这些线性依赖的斜率是“葡萄糖效力”的参数。因此,我们可以将葡萄糖产生和摄取速率表示为:

\begin{equation}\label{2}
    \text{Production} = P_0 -(E_{G0P} + S_{IP} \times I) \times G,
\end{equation}
\begin{equation}\label{3}
    \text{Uptake} = U_0 + (E_{G0U} + S_{IU} \times I) \times G,
\end{equation}

其中$P_0$和$U_0$是零葡萄糖时的葡萄糖产生和摄取速率,\(E_{G0P}\)和\(E_{G0U}\)分别是产生和摄取的零胰岛素葡萄糖效力,\(S_{IP}\)和\(S_{IU}\)分别是产生和摄取的胰岛素敏感性,$I$代表血胰岛素浓度。将方程(\ref{2})和(\ref{3})代入方程(\ref{1}),我们得到

\begin{equation}
    \frac{dG}{dt} = R_0 -(E_{G0} + S_I \times I) \times G,
\end{equation}
其中$R_0$($=P_0-U_0$)是零葡萄糖时葡萄糖的净产生速率,\(E_{G0}\)($=E_{G0p}+E_{G0U}$)是零胰岛素时的总葡萄糖效力,\(S_I\)($=S_{IP}-S_{IU}$)是总胰岛素敏感性\cite{topp2000model}。

再考虑体内的肝糖原水解产生的葡萄糖,最终可以得到
\begin{equation}
    \frac{dG}{dt} = R_0 -(E_{G0} + S_I \times I) \times G+\frac{K_0}{K_1+I^p},
\end{equation}
其中$K_0,K_1,p$为与肝糖原分解产生葡萄糖相关的常数\cite{bridgewater2020amplitude}。
\section{胰岛素动力系统}
胰岛素由胰$\beta$细胞分泌,被肝脏、肾脏和胰岛素受体清除,并分布到几个组分中(例如,门静脉、外周血和间质液)。我们可以通过以下微分方程来描述胰岛素动力系统:

\begin{equation}\label{4}
    \frac{dI}{dt} = \text{Secretion} - \text{Clearance},
\end{equation}
其中Secretion表示胰岛素分泌速率,Clearance表示胰岛素清楚速率。

我们假设胰岛素清除速率为\(kI\),其中$k$是代表肝脏、肾脏和胰岛素受体中胰岛素摄取的清除常数。

当系统处于近稳态时,胰岛素清除速率与血液胰岛素水平成正比。我们假设胰岛素分泌速率模拟为葡萄糖水平的S形函数\cite{topp2000model}。因此,我们假设

\begin{equation}\label{5}
    \text{Secretion} = \frac{\beta\sigma G^2}{(\alpha + G^2)},
\end{equation}

其中$\beta$是胰$\beta$细胞的质量。所有$\beta$细胞被假定以相同的最大速率$\sigma$分泌胰岛素,\(G^2/(\alpha + G^2)\)是一个带有系数$2$的Hill函数,描述了从$0$到$1$的S形范围,在$G=\alpha^{\frac{1}{2}}$时达到其最大值的一半。将方程(\ref{5})代入方程(\ref{4}),我们得到控制胰岛素动力学的方程:

\begin{equation}
    \frac{dI}{dt} = \frac{\beta\sigma G^2}{(\alpha + G^2)} - kI.
\end{equation}

\section{胰\(\beta\)细胞动力系统}
不同的胰$\beta$细胞含量会影响血糖平衡点的位置,在其余参数相同的情况下
\begin{figure}[H]
    \centering
    \includegraphics[width=0.7\textwidth]{Img/nullcline.png}
    \caption{不同胰\(\beta\)细胞下动力系统模型不动点位置的变化}
    \label{fig:nullcline}
\end{figure}
如图\ref{fig:nullcline}所示,当胰$\beta$细胞数量增加时,不动点位置会向右下方移动,即最后平衡时刻血糖含量较低,胰岛素含量较高。

其相图如下图所示
\begin{figure}[H]
    \begin{minipage}[t]{0.5\textwidth}
        \centering
        \includegraphics[width=0.9\textwidth]{Img/phase_300.png}
    \end{minipage}
    \begin{minipage}[t]{0.5\textwidth}
        \centering
        \includegraphics[width=0.9\textwidth]{Img/phase_600.png}
    \end{minipage}
    \caption{不同胰\(\beta\)细胞下动力系统模型相图}
    \label{fig:phase}
\end{figure}

尽管胰$\beta$细胞在胰腺中分布复杂,但$\beta$细胞质量动态可以用单一组分模型定量化,新的$\beta$细胞可以通过现有$\beta$细胞的复制、新生(干细胞的复制和分化)和其他细胞的转分化来形成。目前,无法量化新生和转分化的速率。然而,除了在发育期间和在极端生理或化学诱导创伤反应中,这些可以忽略不计\cite{finegood1995dynamics}。基于这些原因,新生和转分化未纳入当前模型,且生成的$\beta$细胞被假定等于所有复制的$\beta$细胞。

我们可以通过以下微分方程来描述胰$\beta$细胞动力系统:
\begin{equation}
    \frac{d\beta}{dt} = (-d_0+r_1G-r_2G^2)\beta,
\end{equation}

其中$d_0$是零血糖时$\beta$细胞的自然死亡率,$r_1$和$r_2$是两个常系数\cite{topp2000model}。

在此动力系统模型下,我们可以得到胰$\beta$细胞数量与血糖浓度的变化规律。
\begin{figure}[H]
    \centering
    \includegraphics[width=0.7\textwidth]{Img/betarate.png}
    \caption{胰$\beta$细胞生成率和死亡率与血糖浓度的变化规律}
    \label{fig:beta}
\end{figure}

这时整个血糖-胰岛素-胰$\beta$细胞动力系统模型的相图如图\ref{fig:3dphase}所示。
\begin{figure}[H]
    \centering
    \includegraphics[width=0.7\textwidth]{Img/3dphase.png}
    \caption{血糖-胰岛素-胰$\beta$细胞动力系统模型相图}
    \label{fig:3dphase}
\end{figure}

考察胰$\beta$细胞-胰岛素-血糖浓度构成的动力系统
\begin{equation}
    \begin{aligned}
        \dot{G}     & = a_0-a_1G-a_2GI+\frac{a_3}{a_4+I^p}       \\
        \dot{I}     & = \frac{b_1\beta G^2}{G^2 + b_2^2} - b_3 I \\
        \dot{\beta} & = (-d_0+r_1G-r_2G^2)\beta
    \end{aligned}
\end{equation}
我们可以得到如下不动点方程组
\begin{equation}\label{fixequation}
    \begin{aligned}
        a_0-a_1G^*-a_2G^*I^*+\frac{a_3}{a_4+{I^*}^p}     & =0 \\
        \frac{b_1\beta {G^*}^2}{G^{*2} + b_2^2} - b_3I^* & =0 \\
        (-d_0+r_1G^*-r_2{G^*}^2)\beta^*                  & =0
    \end{aligned}
\end{equation}
对于
\begin{equation}\label{G^*}
    -d_0+r_1G^*-r_2{G^*}^2=0
\end{equation}
显然胰$\beta$细胞不会单调递增或递减,因此方程\ref{G^*}一定有两个正实数解,解出的$G^*$,将其带入方程组\ref{fixequation}的第一行我们可以得到对应的$I^*$,再通过第二行可解得对应的$\beta^*$,因此该动力系统模型存在至少两个不动点。对于不动点$(G^*,I^*,\beta^*)$,线性化该动力系统模型可得
\begin{equation}\label{linear}
    \begin{pmatrix}
        \dot{G} \\
        \dot{I} \\
        \dot{\beta}
    \end{pmatrix}=\begin{pmatrix}
        -a_1-a_2I^*                            & -a_2G^*-\frac{pa_3(I^*)^{p-1}}{(a_4+(I^*)^p)^2} & 0                                  \\
        \frac{2b_1\beta^*G^*}{(G^*)^2 + b_2^2} & -b_3                                            & \frac{b_1(G^*)^2}{(G^*)^2 + b_2^2} \\
        r_1\beta^*-2r_2G^*\beta^*              & 0                                               & 0
    \end{pmatrix}\begin{pmatrix}
        G-G^* \\
        I-I^* \\
        \beta-\beta^*
    \end{pmatrix}
\end{equation}

对于方程\ref{G^*}中较小的解$G_1^*$,线性化系统\ref{linear}中的雅可比矩阵的行列式为
\begin{equation}
    \begin{aligned}
        \det&\begin{pmatrix}
            -a_1-a_2I^*                                & -a_2G_1^*-\frac{pa_3(I^*)^{p-1}}{(a_4+(I^*)^p)^2} & 0                                      \\
            \frac{2b_1\beta^*G_1^*}{(G_1^*)^2 + b_2^2} & -b_3                                              & \frac{b_1(G_1^*)^2}{(G_1^*)^2 + b_2^2} \\
            r_1\beta^*-2r_2G_1^*\beta^*                & 0                                                 & 0
        \end{pmatrix}\\
        &=(r_1\beta^*-2r_2G_1^*\beta^*)(-a_2G_1^*-\frac{pa_3(I^*)^{p-1}}{(a_4+(I^*)^p)^2})( \frac{b_1(G_1^*)^2}{(G_1^*)^2 + b_2^2})
    \end{aligned}
\end{equation}
我们有
\begin{equation}
    \begin{aligned}
        r_1\beta^*-2r_2G_1^*\beta^*                       & >0 \\
        -a_2G_1^*-\frac{pa_3(I^*)^{p-1}}{(a_4+(I^*)^p)^2} & <0 \\
        \frac{b_1(G_1^*)^2}{(G_1^*)^2 + b_2^2}            & >0
    \end{aligned}
\end{equation}
因此\begin{equation}
    \det\begin{pmatrix}
        -a_1-a_2I^*                                & -a_2G_1^*-\frac{pa_3(I^*)^{p-1}}{(a_4+(I^*)^p)^2} & 0                                      \\
        \frac{2b_1\beta^*G_1^*}{(G_1^*)^2 + b_2^2} & -b_3                                              & \frac{b_1(G_1^*)^2}{(G_1^*)^2 + b_2^2} \\
        r_1\beta^*-2r_2G_1^*\beta^*                & 0                                                 & 0
    \end{pmatrix}<0,
\end{equation}
此时该不动点为鞍点(saddle)。

对于方程\ref{G^*}中较大的解$G_2^*$,线性化系统\ref{linear}中的雅可比矩阵的行列式为
\begin{equation}
    \begin{aligned}
        \det&\begin{pmatrix}
            -a_1-a_2I^*                                & -a_2G_2^*-\frac{pa_3(I^*)^{p-1}}{(a_4+(I^*)^p)^2} & 0                                      \\
            \frac{2b_1\beta^*G_2^*}{(G_2^*)^2 + b_2^2} & -b_3                                              & \frac{b_1(G_2^*)^2}{(G_2^*)^2 + b_2^2} \\
            r_1\beta^*-2r_2G_2^*\beta^*                & 0                                                 & 0
        \end{pmatrix}\\
        &=(r_1\beta^*-2r_2G_2^*\beta^*)(-a_2G_2^*-\frac{pa_3(I^*)^{p-1}}{(a_4+(I^*)^p)^2})( \frac{b_1(G_2^*)^2}{(G_2^*)^2 + b_2^2})
    \end{aligned}
\end{equation}
我们有
\begin{equation}
    \begin{aligned}
        r_1\beta^*-2r_2G_2^*\beta^*                       & <0 \\
        -a_2G_2^*-\frac{pa_3(I^*)^{p-1}}{(a_4+(I^*)^p)^2} & <0 \\
        \frac{b_1(G_2^*)^2}{(G_2^*)^2 + b_2^2}            & >0
    \end{aligned}
\end{equation}
因此
\begin{equation}
    \det \begin{pmatrix}
        -a_1-a_2I^*                                & -a_2G_2^*-\frac{pa_3(I^*)^{p-1}}{(a_4+(I^*)^p)^2} & 0                                      \\
        \frac{2b_1\beta^*G_2^*}{(G_2^*)^2 + b_2^2} & -b_3                                              & \frac{b_1(G_2^*)^2}{(G_2^*)^2 + b_2^2} \\
        r_1\beta^*-2r_2G_2^*\beta^*                & 0                                                 & 0
    \end{pmatrix}>0,
\end{equation}
且显然
    \begin{equation}
        \text{tr}\begin{pmatrix}
            -a_1-a_2I^*                                & -a_2G_2^*-\frac{pa_3(I^*)^{p-1}}{(a_4+(I^*)^p)^2} & 0                                      \\
            \frac{2b_1\beta^*G_2^*}{(G_2^*)^2 + b_2^2} & -b_3                                              & \frac{b_1(G_2^*)^2}{(G_2^*)^2 + b_2^2} \\
            r_1\beta^*-2r_2G_2^*\beta^*                & 0                                                 & 0
        \end{pmatrix}<0,
    \end{equation}
    因此该不动点为一个稳定不动点。
    
    若我们将$r_1$视为参数,则随着$r_1$的改变会引起不动点性质的变化,当$r_1$较小到满足$r_1^2-4\times r_2\times d_0=0$时,两个不动点重合且这时
    \begin{equation}
        \det \begin{pmatrix}
        -a_1-a_2I^*                            & -a_2G^*-\frac{pa_3(I^*)^{p-1}}{(a_4+(I^*)^p)^2} & 0                                  \\
        \frac{2b_1\beta^*G^*}{(G^*)^2 + b_2^2} & -b_3                                            & \frac{b_1(G^*)^2}{(G^*)^2 + b_2^2} \\
        r_1\beta^*-2r_2G^*\beta^*              & 0                                               & 0
    \end{pmatrix}=0,
    \end{equation}
    当$r_1$再减小时,不动点消失,由定义可得,满足
    \begin{equation}
        r_1^2-4\times r_2\times d_0=0
    \end{equation}
    的$r_1$值为临界值,称为鞍-结分叉(saddle-node bifurcation)。
我们有
\begin{figure}[H]
    \centering
    \includegraphics[width=0.7\textwidth]{Img/betadynamic.png}
    \caption{胰$\beta$细胞动力系统模型分岔}
    \label{fig:bifurcation}
\end{figure}

由于胰$\beta$细胞数量变化较慢(一般以天为单位观测),在实时血糖监测的过程中,我们以每次进食为数据分段标准,因此我们可以考虑将$\beta$细胞视为常量,仅考虑葡萄糖-胰岛素动力系统\cite{huard2022mathematical}。我们可以得到如下动力系统模型:
\begin{equation}\label{11}
    \begin{aligned}
        \dot{G} & = a_0-a_1G-a_2GI+\frac{a_3}{a_4+I^p}  \\
        \dot{I} & = \frac{b_1 G^2}{G^2 + b_2^2} - b_3 I
    \end{aligned}
\end{equation}

对于因为对胰岛素敏感性降低而引起的糖尿病,如图\ref{fig:hyperglycemia}所示,最终血糖的稳定浓度和胰岛素的稳定浓度都高于正常人。
\begin{figure}[H]
    \centering
    \includegraphics[width=0.7\textwidth]{Img/hyperglycemia.png}
    \caption{胰岛素敏感性降低引起的糖尿病}
    \label{fig:hyperglycemia}
\end{figure}

对于常见的二型糖尿病,可能是胰$\beta$细胞数量较少,如图\ref{fig:beta}所示,也可能是胰岛素分泌速率更低,如图\ref{fig:lessinsulin}所示,最终血糖的稳定浓度高于正常人,胰岛素浓度低于正常人,可以通过外界注入胰岛素来使血糖值正常。
\begin{figure}[H]
    \centering
    \includegraphics[width=0.7\textwidth]{Img/lessinsulin.png}
    \caption{胰岛素分泌速率降低引起的糖尿病}
    \label{fig:lessinsulin}
\end{figure}
\section{模型参数估计方法}
在实际应用中,我们需要根据实验数据估计模型参数。在估计模型参数时,我们通常使用最小二乘法来拟合模型。最小二乘法是一种常用的参数估计方法,它通过最小化实际观测值和模型预测值之间的残差平方和来估计模型参数。

\begin{defn}[最小二乘法]
    给定一组实验数据$(x_i, y_i)$,我们的目标是找到一组参数$\theta$,使得模型预测值$f(x_i, \theta)$与实际观测值$y_i$之间的残差平方和最小。最小二乘法的目标函数为:
    \begin{equation}
        \theta^* = \arg\min_{\theta} \sum_{i=1}^{n} (y_i - f(x_i, \theta))^2
    \end{equation}
\end{defn}

在实际应用中,我们通常使用数值优化算法来求解最小二乘问题。常用的数值优化算法包括梯度下降法、共轭梯度法、牛顿法等。这些算法可以有效地求解高维非线性最小二乘问题,帮助我们估计模型参数。

我们先将动力系统模型(\ref{11})转化为差分方程形式,然后根据实验数据使用最小二乘法估计模型参数。具体步骤如下:

\begin{enumerate}
    \item 将动力系统模型(\ref{11})转化为差分方程形式:
          \begin{equation}
              \begin{aligned}
                  G_{t+1} & = G_{t-1} + 2\Delta t \left(a_0-a_1G_t-a_2G_tI_t+\frac{a_3}{a_4+I_t^p}\right) \\
                  I_{t+1} & = I_{t-1} + 2\Delta t \left(\frac{b_1 G_t^2}{G_t^2 + b_2^2} - b_3 I_t\right)
              \end{aligned}
          \end{equation}
    \item 给定一个人一段时间没有外界糖类输入的血糖值向量$(G_i,t_i)_{i=1}^n$,通过差分方程算出来的葡萄糖数值为$f(\theta)$,是一个向量,长度为$n$,对应的是$t_i$时间通过差分方程算出来的血糖值,定义损失函数:
          \begin{equation}
              L(\theta) = \Vert\mathbf{G}-f(\theta) \Vert_2^2+\epsilon\Vert \theta \Vert_2
          \end{equation}
          其中$\epsilon$是正则化参数,用于防止过拟合。
    \item 我们可以使用梯度下降法求解。
          \begin{equation}
              \theta^* = \arg\min_{\theta} L(\theta)
          \end{equation}
\end{enumerate}