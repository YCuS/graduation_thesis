\chapter{引言}\label{chap:introduction}
动力系统模型的发展背景可追溯到20世纪50年代,在控制理论和系统工程领域取得了巨大进展。最初,动力系统模型主要应用于工程和物理系统的建模和控制,如机械系统、电路系统等\cite{hargrove1998dynamic}。随着对生物系统的深入研究和理解,人们开始将动力系统模型应用于生物医学领域,尤其是在糖尿病管理中的应用引起了广泛关注\cite{ellner2006dynamic}。

随着社会的快速发展和人们生活水平的提高,糖尿病已经成为全球性的健康问题之一。据世界卫生组织(WHO)统计,全球约有4.65亿人患有糖尿病,而这一数字还在不断增长\cite{zimmet2016diabetes}。糖尿病是一种慢性疾病,患者需要长期控制血糖水平,以避免发生严重的并发症,如心血管疾病、肾病、视网膜病变等\cite{zheng2018global}。因此,实现对血糖水平的准确预测对于糖尿病患者的管理至关重要。

生理调节的数学模型作为推动科学发展的重要理论工具,在了解生物过程、设计健康调节和疾病标志物、探索健康与非健康状态下的发病机制等方面发挥着重要作用\cite{bakhti2019modelling}。随着可穿戴监测设备的使用增加\cite{kim2020wearable},连续血糖监测(CGM)等技术的发展,我们可以获取日常生活中的更多数据,根据这些数据我们可以将数学模型用于跟踪糖尿病的进展\cite{ha2020type},并制定合理的预测疾病以及治疗策略。

在过去的几十年中,随着计算机技术的迅速发展和生物医学工程领域的不断进步,动力系统模型在实时血糖预测中的应用逐渐引起了人们的关注。动力系统模型可以描述生物系统中各种生理、代谢过程之间的相互作用,并通过数学模型来模拟这些过程的动态变化\cite{cobelli2009diabetes}。血糖调节作为内分泌系统的不可分割部分,其复杂的结果涉及血浆内发生的生化相互作用、肌肉和器官的调节\cite{zavala2019mathematical}。通过对糖尿病患者的生理数据进行监测和分析,结合动力系统模型,可以实现对患者未来一段时间内血糖水平的准确预测,从而指导医生和患者采取合适的治疗策略,提高糖尿病管理的效果。

已有的相关工作主要集中在两个方面:一是基于数学模型的血糖预测方法,二是基于机器学习的预测算法。在数学建模方面,有许多基于生理学的动力系统模型,如Hovorka模型、Bergman最小模型、 Sorencen模型等\cite{pompa2021comparison,mari2001model,de2000mathematical},这些模型可以模拟血糖与胰岛素、饮食、运动等因素之间的动态关系。而在机器学习方面,一些研究者利用大数据技术和深度学习算法,通过分析大量的血糖监测数据,实现了对血糖水平的准确预测。

而在这些已有的模型中,某些模型还有一些局限性,如Hovorka 模型只能描述1型糖尿病患者的葡萄糖代谢情况,并且大部分模型基于的血糖数据都较为离散。因此,本文将研究动力系统模型在实时血糖预测中的应用,通过对动力系统模型的研究和分析,结合实际的血糖监测数据,实现对血糖水平的准确预测。

在本文中首先我们将介绍动力系统模型的基本概念和相关理论,包括动力系统、混沌、不动点、吸引子、极限环等。然后介绍血糖动力系统中的一些常见模型即其原理,并且对这些模型进行相应的模型分析和算法分析,之后利用收集的数据进行模型拟合,比较分析各模型的优缺点,用拟合的模型进行血糖预测,尽可能接近真实血糖数据。