\chapter{引言}\label{chap:introduction}
\section{动力系统发展}
动力系统最初是作为物理学的一个分支出现的,特别是作为经典力学的一部分来研究物体的运动和力的相互作用。这一领域的研究可以追溯到17世纪末和18世纪初,由牛顿的经典力学奠定基础。在这一时期,动力系统主要研究物体在给定条件下的运动和轨迹,通过微分方程描述物体在力作用下的运动状态。

然而,动力系统逐渐发展成一门数学的交叉学科。在20世纪初,亨利·庞加莱(Henri Poincaré)等数学家开始探索动力系统中非线性方程的复杂行为。庞加莱研究了行星轨道的稳定性,发现了一些复杂的运动模式,例如混沌和周期运动,这些研究拓展了动力系统的理论框架。

随后,俄罗斯数学家李亚普诺夫发展了动力系统稳定性的现代理论,他在1899年提出的方法使得定义常微分方程组的稳定性成为可能。再到之后,随着基础数学的发展,一些基础数学其他方向上的理论也被引入到动力系统中,如拓扑学、几何学,遍历理论等。这些理论的引入使得动力系统的研究更加丰富和深入。

在任何给定时间,动力系统都有一个状态,表示适当状态空间中的一个点。这种状态通常由实数元组或几何流形中的向量给出。动力系统的演化规则是一个函数,它描述了未来状态与当前状态的结合。通常,该函数是确定性的,也就是说,在给定的时间间隔内,当前状态只遵循一个未来状态\cite{enwiki:1216667016}。这使得我们拥有了描述一些随着时间变化的系统的行为的方法。然而后续发现,有些系统的演化规则是不确定的,即使在确定的演化规则下,有的动力系统初始值有微小的变化会引起整个系统大不相同,著名数学家 Vladimir Arnold 举过一个著名的例子,他用一张猫的图片展示了$\mathbb{T}^2$上的双曲自同构混沌映射。我们将映射离散化,对于长度为$N$的一个环,该映射可写为一个二阶离散动力系统
\begin{equation}
    x_{n+1}-3x_n+x_{n-1}=0 \mod N
\end{equation}
令$y_{n+1}=x_{n+1}-x_n$,则可将其写为一阶离散动力系统
\begin{equation}
    \begin{aligned}
        x_{n+1}&=2x_n+y_n \mod N\\
        y_{n+1}&=x_{n+1}-x_n \mod N
    \end{aligned}
\end{equation}
可以证明在至多$3N$次迭代后可以得到原始图片\cite{enwiki:1223508209}。

如图\ref{fig:arnold_cat_map},通过这个演化规则,我们可以看到,图片在不断的变化,通过拆分,重组,在同一个演化规则下,整张图片从有序变为了无序,最后又变为了有序。

\begin{figure}[H]
    \centering
    \includegraphics[width=\textwidth]{Img/arnold.png}
    \caption{Arnold猫图(上科大校徽版)}
    \label{fig:arnold_cat_map}
\end{figure}

\section{动力系统模型的发展}
随着时间的推移,动力系统逐渐扩展到许多其他学科,例如生物学、化学、工程学和经济学等领域。它们都借鉴了动力系统的理论和方法来分析系统的动态行为和稳定性。

如今,动力系统作为一门交叉学科,它融合了物理学和数学的方法,特别是微分方程、线性代数、拓扑学和数值分析等。在这方面,它不仅研究连续和离散动力系统的行为,还研究混沌理论、稳定性分析、吸引子和分叉等复杂现象。因此,动力系统不仅在理论上继续发展,还在实际应用中发挥着重要作用。

动力系统模型的发展背景可追溯到20世纪50年代,在控制理论和系统工程领域取得了巨大进展。最初,动力系统模型主要应用于工程和物理系统的建模和控制,如机械系统、电路系统等\cite{hargrove1998dynamic}。随着对生物系统的深入研究和理解,人们开始将动力系统模型应用于生物医学领域,尤其是在糖尿病管理中的应用引起了广泛关注\cite{ellner2006dynamic}。

随着社会的快速发展和人们生活水平的提高,糖尿病已经成为全球性的健康问题之一。据世界卫生组织(WHO)统计,全球约有4.65亿人患有糖尿病,而这一数字还在不断增长\cite{zimmet2016diabetes}。糖尿病是一种慢性疾病,患者需要长期控制血糖水平,以避免发生严重的并发症,如心血管疾病、肾病、视网膜病变等\cite{zheng2018global}。因此,实现对血糖水平的准确预测对于糖尿病患者的管理至关重要。

生理调节的数学模型作为推动科学发展的重要理论工具,在了解生物过程、设计健康调节和疾病标志物、探索健康与非健康状态下的发病机制等方面发挥着重要作用\cite{bakhti2019modelling}。随着可穿戴监测设备的使用增加\cite{kim2020wearable},连续血糖监测(CGM)等技术的发展,我们可以获取日常生活中的更多数据,根据这些数据我们可以将数学模型用于跟踪糖尿病的进展\cite{ha2020type},并制定合理的预测疾病以及治疗策略。

在过去的几十年中,随着计算机技术的迅速发展和生物医学工程领域的不断进步,动力系统模型在实时血糖预测中的应用逐渐引起了人们的关注。动力系统模型可以描述生物系统中各种生理、代谢过程之间的相互作用,并通过数学模型来模拟这些过程的动态变化\cite{cobelli2009diabetes}。血糖调节作为内分泌系统的不可分割部分,其复杂的结果涉及血浆内发生的生化相互作用、肌肉和器官的调节\cite{zavala2019mathematical}。通过对糖尿病患者的生理数据进行监测和分析,结合动力系统模型,可以实现对患者未来一段时间内血糖水平的准确预测,从而指导医生和患者采取合适的治疗策略,提高糖尿病管理的效果。

已有的相关工作主要集中在两个方面:一是基于数学模型的血糖预测方法,二是基于机器学习的预测算法。在数学建模方面,有许多基于生理学的动力系统模型,如Hovorka模型、Bergman最小模型、 Sorencen模型等\cite{pompa2021comparison,mari2001model,de2000mathematical},这些模型可以模拟血糖与胰岛素、饮食、运动等因素之间的动态关系。而在机器学习方面,一些研究者利用大数据技术和深度学习算法,通过分析大量的血糖监测数据,实现了对血糖水平的准确预测。

而在这些已有的模型中,某些模型还有一些局限性,如Hovorka 模型只能描述1型糖尿病患者的葡萄糖代谢情况,并且大部分模型基于的血糖数据都较为离散。因此,本文将研究动力系统模型在实时血糖预测中的应用,通过对动力系统模型的研究和分析,结合实际的血糖监测数据,实现对血糖水平的准确预测。

在本文中首先我们将介绍动力系统模型的基本概念和相关理论,包括动力系统、混沌、不动点、吸引子、极限环等。然后介绍血糖动力系统中的一些常见模型即其原理,并且对这些模型进行相应的模型分析和算法分析,之后利用收集的数据进行模型拟合,用拟合的模型进行血糖预测,尽可能接近真实血糖数据。

最后根据模型拟合的参数,我们可以对病人的病情以及治疗方案进行推测,为病人提供个性化的治疗方案,提高糖尿病管理的效果。