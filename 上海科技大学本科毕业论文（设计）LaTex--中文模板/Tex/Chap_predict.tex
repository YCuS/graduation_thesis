\chapter{实时血糖预测}\label{chap:predict}
\section{实时血糖监测技术}
实时血糖监测(Continuous Glucose Monitoring, CGM)技术是一种通过连续监测血液中葡萄糖浓度的方法,以实时、动态地监测糖尿病患者的血糖水平。与传统的间断性血糖检测相比,CGM技术能够提供更加精确和详细的血糖数据,帮助患者更好地了解自己的血糖波动情况,并及时调整胰岛素剂量或饮食习惯。

CGM技术通常由一个植入皮肤下的葡萄糖传感器和一个便携式的数据接收器组成。传感器定期测量组织液中的葡萄糖浓度,并将数据传输到接收器上。接收器可以显示实时血糖数据,或者将数据传输到手机或计算机上进行进一步分析和记录。

通过CGM技术,糖尿病患者可以监测到血糖的波动情况,及时发现低血糖或高血糖的风险,并采取相应的措施进行调整。此外,CGM技术还可以提供血糖趋势预测和报警功能,帮助患者更好地管理血糖\cite{vigersky2017role}。

近年来,随着CGM技术的不断发展和改进,越来越多的糖尿病患者选择使用CGM技术来管理他们的血糖。CGM技术的广泛应用为糖尿病管理提供了新的思路和方法,有望进一步改善患者的生活质量和健康状况。
\section{数据处理}
从我们收集到的病人血糖数据中可以发现,病人服用与胰岛素有关的药物以及进食一般是同时进行的,因此可以将进食视为以此扰动,以每次进食为界线,可以将一个病人的实时血糖数据分为$(G_{ij}, t_{ij})$,其中$j$表示该病人数据记录的第j次进食。

同时,我们也要衡量具体扰动的大小,由于我们只知道病人吃了什么,并没有确切的葡萄糖和胰岛素变化数值,因此我们使用薄荷健康的数据,将病人每次的进食数据转换为卡路里,葡萄糖等数据,再配上服用药物中含有的胰岛素等数据,以此作为扰动的大小。
\section{动力系统模型的选择以及训练}
\section{实时血糖预测的算法和模型评估} 