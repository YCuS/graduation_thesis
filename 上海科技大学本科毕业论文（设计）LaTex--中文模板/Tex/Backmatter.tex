%\chapter{作者简历及攻读学位期间发表的学术论文与研究成果}
%
%\textbf{本科生无需此部分}. 
%
%\section*{作者简历}
%
%\subsection*{casthesis作者}
%
%吴凌云,福建省屏南县人,中国科学院数学与系统科学研究院博士研究生. 
%
%\subsection*{ucasthesis作者}
%
%莫晃锐,湖南省湘潭县人,中国科学院力学研究所硕士研究生. 
%
%\section*{已发表(或正式接受)的学术论文:}
%
%[1] ucasthesis: A LaTeX Thesis Template for the University of Chinese Academy of Sciences, 2014.
%
%\section*{申请或已获得的专利:}
%
%(无专利时此项不必列出)
%
%\section*{参加的研究项目及获奖情况:}
%
%可以随意添加新的条目或是结构. 

\chapter[致谢]{致\quad 谢}\chaptermark{致\quad 谢}% syntax: \chapter[目录]{标题}\chaptermark{页眉}
四年的本科生活即将结束,回首这四年,我感慨万千. 在这里,感谢葛淑菲老师在我本科毕业设计中的指导,感谢她在我学习和生活中的关心和帮助. 感谢李雷老师指导我学习了解了系统生物学相关知识,感谢他在我学习和科研中的指导和帮助. 感谢张正老师和戈振超老师,它们帮助我在大一大二迷茫时选择了合适的方向,并且对我的数学学习提供了莫大的帮助. 

感谢我的同学和朋友,是他们在我遇到困难时给予我帮助和支持,让我感受到了友情的温暖. 感谢上海科技大学,是这里给予了我一个学习和成长的平台,让我收获了知识和友情. 感谢所有关心和帮助过我的人,是你们让我在这四年中变得更加坚强和成熟. 你们在我生活中的陪伴和支持让我感到无比幸运和快乐. 祝愿所有的朋友们前程似锦,一帆风顺. 

最后,更要感谢我的父母,是他们在我在我背后一直支持着我,让我坚持走下去. 他们在我生活上的关心与鼓励让我有更多时间投入到科研与学习中. 
%\thispagestyle{noheaderstyle}% 如果需要移除当前页的页眉
%\pagestyle{noheaderstyle}% 如果需要移除整章的页眉
\pagestyle{mainmatterstyle} % 与前文页眉页脚格式相同
\cleardoublepage[plain]% 让文档总是结束于偶数页,可根据需要设定页眉页脚样式,如 [noheaderstyle]

