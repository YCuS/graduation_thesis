%---------------------------------------------------------------------------%
%->> 封面信息及生成
%---------------------------------------------------------------------------%
%-
%-> 中文封面信息
%-
\confidential{}% 密级:只有涉密论文才填写
\schoollogo{scale=0.2}{ShanghaiTechLogo}% 校徽
%======论文题目 页眉显示 务必准确========
\title{动力系统模型及其在实时血糖预测的应用}% 论文中文题目
\author{刘益通}% 论文作者
\ID{2020511013}
\entranceYear{2020}
\advisor{葛淑菲}% 指导教师:姓名
\advisorsec{}% 第二指导老师:按情况填写
\degree{学士}% 学位:学士、硕士、博士
\degreetype{理学}% 学位类别:理学、工学、工程、医学等
\major{数学与应用数学}% 二级学科专业名称
\institute{上海科技大学数学科学研究所}% 院系名称
\chinesedate{二零二四年~6月}% 毕业日期:夏季为6月、冬季为12月
%-
%-> 英文封面信息
%-
\englishtitle{A Brilliant Work}% 论文英文题目
\englishauthor{Liu Yitong}% 论文作者
\englishadvisor{Supervisor: Ge Shufei}% 指导教师
\englishdegree{Bachelor}% 学位:Bachelor, Master, Doctor. 封面格式将根据英文学位名称自动切换,请确保拼写准确无误
\englishdegreetype{Philosophy}% 学位类别:Philosophy, Natural Science, Engineering, Economics, Agriculture 等
\englishthesistype{thesis}% 论文类型: thesis, dissertation
\englishmajor{English Philosophy}% 二级学科专业名称
\englishinstitute{School of Foreign Language}% 院系名称
\englishdate{\quad / \quad /\quad}% 毕业日期:夏季为June、冬季为December
%-
%-> 生成封面
%-
%-======================================================
% 封面和声明页格式不符合教务处要求,因此此处不生成pdf格式的 ||
% 论文封面和声明页,请各位使用word模板中的相关内容.        ||
%=======================================================
%\maketitle% 生成中文封面
%\makeenglishtitle% 生成英文封面
%-
%-> 作者声明
%-
%\makedeclaration% 生成声明页
%-
%-> 中文摘要
%-

\chapter*{摘\quad 要}\chaptermark{摘\quad 要}% 摘要标题
\setcounter{page}{1}% 开始页码
\pagenumbering{Roman}% 页码符号

本文旨在探讨动力系统在生物医学领域的应用,特别是在糖尿病治疗和管理中的作用. 本文首先简要介绍了动力系统的基础知识,以及一些与动力系统性质分析有关的定理,如庞加莱-本迪克松定理. 同时本文还给出了一些示例来直观地展示动力系统,为理解后续内容及模型稳定性分析奠定了基础. 此外,本文详细阐述了葡萄糖、胰岛素和胰$\beta$细胞的动力系统模型. 本文将连续血糖监测技术采集的数据用于动力系统模型的拟合以精确模拟患者的血糖水平变化. 最后,本文对模型训练结果进行了分析,验证了模型的准确性和实用性. 模型发现我们可以根据估计的参数衡量病人糖尿病的类型,且可以根据拟合的模型预测近期一段时间的病人进食后的血糖实时变化曲线. 


\keywords{动力系统,连续血糖监测,血糖预测,葡萄糖-胰岛素系统}% 中文关键词
%-
%-> 英文摘要
%-
\chapter*{Abstract}\chaptermark{Abstract}% 摘要标题

This thesis aims to explore the application of dynamical systems in the biomedical field, particularly in the treatment and management of diabetes.  The thesis commences with an overview of the fundamental concepts of dynamical systems, including pivotal theorems such as the Poincaré-Bendixson theorem that are essential for analyzing the characteristics of dynamical systems. It also provides illustrative examples to visually represent dynamical systems, which lays the groundwork for comprehending the subsequent material and conducting stability analyses of the models. In addition, the thesis meticulously details the dynamical system models pertaining to glucose, insulin, and pancreatic-$\beta$
cells. It utilizes data acquired from Continuous Glucose Monitoring (CGM) technology to calibrate the dynamical system model, thereby precisely emulating the fluctuations in a patient's blood glucose levels. Ultimately, the thesis evaluates the training outcomes of the model, affirming its accuracy and utility. It discovers that the estimated parameters of the model can approximate the type of diabetes a patient has, and the calibrated model can forecast the real-time blood glucose trajectory post-meal consumption over an imminent short duration.

\englishkeywords{Dynamic System, Continuous Glucose Monitoring, Glucose prediction, Glucose-insulin model}% 英文关键词
%---------------------------------------------------------------------------%
